%!TEX options = shell-escape
% ^ special comment for svg to load

% format
\documentclass [10pt, a4paper] {extarticle}

% page margins
\usepackage [left=0.6in, right=0.7in, top=0.6in, bottom=0.6in] {geometry}

% draw lines
\usepackage {tikz}

% hyperlinks
\usepackage {hyperref}

% lorem ipsum
\usepackage {lipsum}

% bullet lists configs
\usepackage {enumitem}

% icons
\usepackage {fontawesome5}
\usepackage {wasysym}

% qr code
\usepackage {svg}

% no page number
\thispagestyle {empty}

%%%%%%%%%%%%%%%%%%%%%%
%%%%%%% NAME %%%%%%%%%
%%%%%%%%%%%%%%%%%%%%%%

% colours
\definecolor {namelines} {HTML} { 333333 }
\definecolor {nametoptxt} {HTML} { 7D92A1 }
\definecolor {namebottomtxt} {HTML} { 333333 }
\definecolor {nametoptxt} {HTML} { 7D92A1 }

\newcommand{\sectionlinethickness} {1.3 pt }

% section size
\newcommand{\leftcolumwidth} {0.18 }
\newcommand{\rightcolumwidth}{ 0.82 }

\newcommand {\sectiontitle}[1] {
    \begin {flushleft}
    \begin {minipage}[c]{\leftcolumwidth\textwidth}
        \begin {flushright}
        \!\MakeUppercase {#1}
        \hspace* {10px}
        \end {flushright}
    \end {minipage}
        \begin {tikzpicture}
            \hspace{-4px}
            \draw [line width=\sectionlinethickness, namelines] (1,0) -- (15.363,0);
        \end {tikzpicture}
    \end {flushleft}
}


\begin{document}

    %%%%%%%%%%%%%%%%%%%%%%
    %%%%%%% NAME %%%%%%%%%
    %%%%%%%%%%%%%%%%%%%%%%

    \begin {flushright}
    \begin {minipage} [t] {\rightcolumwidth\textwidth}
        \vspace {0.35\baselineskip}
        \begin {minipage}[t]{0.45\textwidth}

            \begin {tikzpicture}
                \draw [line width=\sectionlinethickness, namelines] (0,1) -- (0,1.5);
            \end {tikzpicture}

            \vspace {0.2cm}

            {\fontsize {23 pt} {23 pt}{
                \color{nametoptxt} \bfseries
                \!\!\MakeUppercase {yago}
            }}

            \vspace{0.3cm}

            {\fontsize {15 pt} {15 pt}{
                \color{namebottomtxt}
                \!\MakeUppercase {fernández rego}
            }}

            \vspace {0.2cm}

            \begin {tikzpicture}
                \draw [line width=\sectionlinethickness, namelines] (0,1) -- (0,1.75);
            \end {tikzpicture}
        \end {minipage}
        \hfill
        \vrule
        \hspace*{5pt}
        \begin {minipage}[t]{0.375\textwidth}

            \vspace*{\fill}
            \begin {center}
            \begin {tabular}{cl}
                \faMapMarker & A Coruña, Galicia, España
                \\[5pt] \faEnvelope  & \href {mailto:yago.fernandez.rego@udc.es} {yago.fernandez.rego@udc.es}
                \\[5pt] \faMobile    & +34 698135841
                \\[5pt] \faGithub    & \href {https://github.com/yref-boop}{yref-boop}
            \end {tabular}
            \end {center}
            \vspace*{\fill}

        \end {minipage}

    \end {minipage}
    \end {flushright}


    %%%%%%%%%%%%%%%%%%%%%%
    %%%%% ABOUT ME %%%%%%%
    %%%%%%%%%%%%%%%%%%%%%%

    % title & separator
    \sectiontitle {sobre mí}

    % section contents
    \begin {flushright} \begin {minipage} [t] {\rightcolumwidth\textwidth}
        Estudiante de Ingeniería Informática de 21 años. Disfruto resolviendo problemas así como intentando entender tanto el funcionamiento interno de sistemas complejos como conceptos teóricos, al igual que colaborar con otros. \\ [3px]
        Busco un puesto donde pueda crecer y aprender de compañeros más experimentados, mientras aporto experiencia en proyectos ya ejecutados.
    \end {minipage}
    \end {flushright}

    %%%%%%%%%%%%%%%%%%%%%%
    %%%%% EDUCATION %%%%%%
    %%%%%%%%%%%%%%%%%%%%%%

    % title & separator
    \sectiontitle {educación}

    % section contents
    \begin {flushright}
    \begin {minipage} [t] {\rightcolumwidth\textwidth}
        \textbf {Universidade da Coruña, Facultade de Informática da Coruña}\\
        \href {https://estudos.udc.es/en/study/start/614G01V01} {\textbf {Grado Bilingüe (Inglés) en Ingeniería Informática}}\\
        \hspace* {6px} \textcolor{namebottomtxt} {Septiembre 2020 - Actualidad} \\ [4px]
        Especializado en Ciencias de la Computación. \\ [4px]
        Asignaturas relevantes: Matemática Discreta, Algoritmos, Paradigmas de la Programación, Apredizaje Automático, Desarrollo de Sistemas Inteligentes, Diseño Software.
    \end {minipage}
    \end {flushright}

    %%%%%%%%%%%%%%%%%%%%%%
    %%%%% PROJECTS %%%%%%
    %%%%%%%%%%%%%%%%%%%%%%

    % title & separator
    \sectiontitle {proyectos}

    % section contents
    \begin {flushright}
    \begin {minipage} [t] {\rightcolumwidth\textwidth}
        A pesar de no tener experiencia profesional, he participado en numerosos proyectos académicos, que me han ayudado a desarrollar habilidades de comunicación, organización y resolución de problemas, así como familiaridad con diferentes tecnologías. \\ [7px]
        \begin {minipage} [t] {0.3\textwidth}
            \href {https://github.com/yref-boop/automatic-learning} {\textbf {Aprendizaje Automático}} \\ [-5px]
            \hrule
            \vspace {4px} Diseño e implementación de un sistema inteligente para el reconocimiento de diversas frutas en imágenes.
            \vspace {-5px}
            \begin {itemize} [noitemsep]
                \item tamaño equipo: 4
                \item lenguaje: Julia
            \end {itemize}
        \end {minipage}
        \hfill
        \begin {minipage} [t] {0.3\textwidth}
            \href {https://github.com/yref-boop/ann} {\textbf {Red Neuronal Artificial}} \\ [-5px]
            \hrule
            \vspace {4px} Implementación de una Red de Neuronas Artificiales evitando el uso de librerías externas.
            \vspace {-5px}
            \begin {itemize} [noitemsep]
                \item proyecto personal
                \item lenguaje: Haskell
            \end {itemize}
        \end {minipage}
        \hfill
        \begin {minipage} [t] {0.3\textwidth}
            \href {https://github.com/yref-boop/distributed-systems}{\textbf {Sistemas Distribuidos}} \\ [-5px]
            \hrule
            \vspace {4px} Implementación utilizando arquitectura por capas de una aplicación de gestión de eventos
            \vspace {-5px}
            \begin {itemize} [noitemsep]
                \item tamaño equipo: 3
                \item lenguaje: Java
            \end {itemize}
        \end {minipage}
    \end {minipage}
    \end {flushright}

    %%%%%%%%%%%%%%%%%%%%%%
    %%%%%%% SKILLS %%%%%%%
    %%%%%%%%%%%%%%%%%%%%%%

    % title & separator
    \sectiontitle {aptitudes}

    % section contents
    \begin {flushright}
    \begin {minipage} [t] {\rightcolumwidth\textwidth}

        \textbf {Habilidades Técnicas} \\ [4px]
        Me considero una persona flexible en lo referente a nuevas tecnologías. Actualmente me centro principalmente en conseguir habilidades generales como lógica, estructuras de datos, eficiencia, paradigmas... para poder aprender de forma fácil y rápida tecnologías y lenguajes nuevos.
        \vspace {-2px}
        \begin {itemize} [noitemsep]
            \item Herramientas: JetBrains Suite, Neovim, Bash, LaTeX, Git
            \item Lenguajes: C, Haskell, Ocaml, Python, Java, Julia
        \end {itemize}

        \begin {minipage} [t] {0.475\textwidth}
            \textbf {Intereses} \\ [4px]
            Existen muchos elementos comunes entre mi perfil técnico y mis intereses personales:
            \vspace {-3px}
            \begin {itemize} [noitemsep]
                \item Programación Funcional
                \item Inteligencia Artificial
                \item Avances Académicos
                \item Open-Source
                \item Últimas Tecnologías
            \end {itemize}
        \end {minipage}
        \hfill
        \begin {minipage} [t] {0.475\textwidth}
            \textbf {Soft Skills} \\ [4px]
            Además de mi stack técnico, traigo conmigo las siguientes habilidades interpersonales:
            \vspace {-3px}
            \begin {itemize} [noitemsep]
                \item Adaptabilidad
                \item Colaboración y Trabajo en Equipo
                \item Resolución de Problemas
                \item Independencia
                \item Curiosidad
            \end {itemize}
        \end {minipage}
    \end {minipage}
    \end {flushright}

    %%%%%%%%%%%%%%%%%%%%%%
    %%%%% LANGUAGES %%%%%%
    %%%%%%%%%%%%%%%%%%%%%%

    % title & separator
    \begin {flushleft}
    \begin {minipage}[c]{\leftcolumwidth\textwidth}
        \begin {flushright}
        \!\MakeUppercase {idiomas}
        \hspace* {10px}
        \end {flushright}
    \end {minipage}
        \begin {tikzpicture}
            \hspace{-4px}
            \draw [line width=\sectionlinethickness, namelines] (1,0) -- (10.1,0);
        \end {tikzpicture}
    \end {flushleft}


    % section contents
    \begin {flushright}
    \begin {minipage} [t] {\rightcolumwidth\textwidth}
        \begin {minipage} [t] {0.7\textwidth}
            \begin {tabular} {llcccccc}
                Idioma & Acreditación & A1 & A2 & B1 & B2 & C1 & C2 \\[2px]
                \hline \\[-9px]
                Español & Nativo & \newmoon & \newmoon & \newmoon & \newmoon & \newmoon & \newmoon \\
                Gallego & Nativo & \newmoon & \newmoon & \newmoon & \newmoon & \newmoon & \newmoon \\
                Inglés & Cambridge & \newmoon & \newmoon & \newmoon & \newmoon & \newmoon & \fullmoon \\
                Francés & DELF & \newmoon & \newmoon & \newmoon & \fullmoon & \fullmoon & \fullmoon \\
            \end{tabular}
        \end {minipage}
        \hfill
        \begin {minipage} [t] {0.2\textwidth}
        \vspace* {-30px}
            \includesvg [width=65px, height=65px] {qr}
        \hspace* {12px}
        \end {minipage}
    \end {minipage}
    \end {flushright}


\end{document}

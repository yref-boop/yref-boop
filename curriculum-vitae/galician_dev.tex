%! TEX program = pdf_escaped
% ^ special comment for svg to load

% format
\documentclass [10pt, a4paper] {extarticle}

% page margins
\usepackage [left=0.6in, right=0.7in, top=0.6in, bottom=0.6in] {geometry}

% draw lines
\usepackage {tikz}

% hyperlinks
\usepackage {hyperref}

% lorem ipsum
\usepackage {lipsum}

% bullet lists configs
\usepackage {enumitem}

% icons
\usepackage {fontawesome5}
\usepackage {wasysym}

% qr code
\usepackage {svg}

% no page number
 \thispagestyle{empty}

%%%%%%%%%%%%%%%%%%%%%%
%%%%%%% NAME %%%%%%%%%
%%%%%%%%%%%%%%%%%%%%%%

% colours
\definecolor {namelines} {HTML} { 333333 }
\definecolor {nametoptxt} {HTML}{ 7D92A1 }
\definecolor {namebottomtxt} {HTML} { 333333 }
\definecolor {nametoptxt} {HTML} { 7D92A1 }

\newcommand{\sectionlinethickness} {1.3 pt }

% section size
\newcommand{\leftcolumwidth} {0.18 }
\newcommand{\rightcolumwidth}{ 0.82 }

\newcommand {\sectiontitle}[1] {
    \begin {flushleft}
    \begin {minipage}[c]{\leftcolumwidth\textwidth}
        \begin {flushright}
        \!\MakeUppercase {#1}
        \hspace* {10px}
        \end {flushright}
    \end {minipage}
        \begin {tikzpicture}
            \hspace{-4px}
            \draw [line width=\sectionlinethickness, namelines] (1,0) -- (15.363,0);
        \end {tikzpicture}
    \end {flushleft}
}


\begin{document}

    %%%%%%%%%%%%%%%%%%%%%%
    %%%%%%% NAME %%%%%%%%%
    %%%%%%%%%%%%%%%%%%%%%%

    \begin {flushright}
    \begin {minipage} [t] {\rightcolumwidth\textwidth}
        \vspace {0.35\baselineskip}
        \begin {minipage}[t]{0.45\textwidth}

            \begin {tikzpicture}
                \draw [line width=\sectionlinethickness, namelines] (0,1) -- (0,1.5);
            \end {tikzpicture}

            \vspace {0.2cm}

            {\fontsize {23 pt} {23 pt}{
                \color{nametoptxt} \bfseries
                \!\!\MakeUppercase {yago}
            }}

            \vspace{0.3cm}

            {\fontsize {15 pt} {15 pt}{
                \color{namebottomtxt}
                \!\MakeUppercase {Fernández Rego}
            }}

            \vspace {0.2cm}

            {\fontsize {15 pt} {15 pt}{
                \color{namebottomtxt}
                \!\MakeUppercase {desenvolvedor software}
            }}

            \vspace{0.2cm}

            \begin {tikzpicture}
                \draw [line width=\sectionlinethickness, namelines] (0,1) -- (0,1.5);
            \end {tikzpicture}
        \end {minipage}
        \hfill
        \vrule
        \hspace*{5pt}
        \begin {minipage}[t]{0.375\textwidth}

            \vspace*{\fill}
            \begin {center}
            \begin {tabular}{cl}
                \faMapMarker & A Coruña, Galicia, Spain
                \\[5pt] \faEnvelope  & \href {mailto:yago.fdez.rego@gmail.com} {yago.fdez.rego@gmail.com}
                \\[5pt] \faMobile    & +34 698135841
                \\[5pt] \faGithub    & \href {https://github.com/yref-boop}{yref-boop}
            \end {tabular}
            \end {center}
            \vspace*{\fill}

        \end {minipage}

    \end {minipage}
    \end {flushright}


    %%%%%%%%%%%%%%%%%%%%%%
    %%%%% ABOUT ME %%%%%%%
    %%%%%%%%%%%%%%%%%%%%%%

    % title & separator
    \sectiontitle {sobre min}

    % section contents
    \begin {flushright} \begin {minipage} [t] {\rightcolumwidth\textwidth}
        Estudante de Enxeñaría Informática de vinte anos aspirante a ser desenvolvedor software. Desfruto da resolución de problemas así coma entender o funcionamento interno de sistemas complexos ou concetos teóricos, así coma colaborar con outros \\[3px]
        Busco un posto onde poida crecer e aprender de compañeiros máis experimentados, mentres aporto experiencia en proxectos que xa executei.
    \end {minipage}
    \end {flushright}

    %%%%%%%%%%%%%%%%%%%%%%
    %%%%% EDUCATION %%%%%%
    %%%%%%%%%%%%%%%%%%%%%%

    % title & separator
    \sectiontitle {educación}

    % section contents
    \begin {flushright}
    \begin {minipage} [t] {\rightcolumwidth\textwidth}
        \textbf {Universidade da Coruña, Facultade de Informática da Coruña}\\
        \href {https://estudos.udc.es/en/study/start/614G01V01} {\textbf {Grao Bilingüe (Esp, Ing) en Enxeñaría Informática}}\\
        \hspace* {6px} \textcolor{namebottomtxt} {Setembro 2020 - Actualidade} \\ [4px]
        Especialización en Ciencias da Computación \\ [4px]
        Materias relevantes: Matemática Discreta, Algoritmos, Paradigmas da Programación, Aprendizaxe Automática, Desenvolvemento de Sistemas Intelixentes, Deseño de Software.
    \end {minipage}
    \end {flushright}

    %%%%%%%%%%%%%%%%%%%%%%
    %%%%% PROJECTS %%%%%%
    %%%%%%%%%%%%%%%%%%%%%%

    % title & separator
    \sectiontitle {proxectos}

    % section contents
    \begin {flushright}
    \begin {minipage} [t] {\rightcolumwidth\textwidth}
        Malia ter falta de experiencia profesional, teño participado en numerosos proxectos académicos, nos que desenvolvín habilidades de comunicación, solución de problemas e organización, así coma familiariade con distintas tecnoloxías \\ [7px]
        \begin {minipage} [t] {0.3\textwidth}
            \textbf {Algorithm Analysis} \\ [-5px]
            \hrule
            \hspace* {2px} Análise teórica e empírica da complexidade algorítimica de varios algoritmos de ordenamento.
            \vspace {-5px}
            \begin {itemize} [noitemsep]
            \item tamaño do equipo: 3
            \item linguaxe: C
            \end {itemize}
        \end {minipage}
        \hfill
        \begin {minipage} [t] {0.3\textwidth}
            \href {https://github.com/yref-boop/ann} {\textbf {Neural Network}} \\ [-5px]
            \hrule
            \vspace {4px} Implementación en Haskell dunha Red de Neuronas Artificiais sen uso de librerías xa programadas.
            \vspace {-5px}
                \begin {itemize} [noitemsep]
                \item proxecto independente
                \item linguaxe: Haskell
            \end {itemize}
        \end {minipage}
        \hfill
        \begin {minipage} [t] {0.3\textwidth}
            \textbf {Distributed Systems} \\ [-5px]
            \hrule
            \vspace {4px} Implementación utilizando arquitectura por capas dunha aplicación de xestión de eventos.
            \vspace {-5px}
            \begin {itemize} [noitemsep]
                \item tamaño do equipo: 3
                \item linguaxe: Java
            \end {itemize}
        \end {minipage}
    \end {minipage}
    \end {flushright}

    %%%%%%%%%%%%%%%%%%%%%%
    %%%%%%% SKILLS %%%%%%%
    %%%%%%%%%%%%%%%%%%%%%%

    % title & separator
    \sectiontitle {aptitudes}

    % section contents
    \begin {flushright}
    \begin {minipage} [t] {\rightcolumwidth\textwidth}

        \textbf {Habilidades Técnicas} \\ [4px]
        Considérome relativemente flexible con respecto novas tecnoloxías. Actualmente céntrome en desenvolver habilidades xerais coma lóxica, estruturas de datos, eficiencia, paradigmas... Para poder aprender rápida e doadamente tecnoloxías e linguaxes novas.
        \vspace {-2px}
        \begin {itemize} [noitemsep]
            \item Ferramentas: JetBrains Suite, Neovim, Bash, LaTeX, Git
            \item Linguaxes de programación: C, Haskell, Ocaml, Python, Java
        \end {itemize}

        \begin {minipage} [t] {0.475\textwidth}
            \textbf {Intereses} \\ [4px]
            Existen moitos elementos comúns entre o meu perfil técnico es os meus intereses persoais:
            \vspace {-3px}
            \begin {itemize} [noitemsep]
                \item Programación Funcional
                \item Intelixencia Artificial
                \item Avances Académicos
                \item Open-Source
                \item Últimas Tecnoloxías
            \end {itemize}
        \end {minipage}
        \hfill
        \begin {minipage} [t] {0.475\textwidth}
            \textbf {Soft Skills} \\ [4px]
            Ademais do meu stack técnico, traio conmigo as seguintes habilidades interpersoais:
            \vspace {-3px}
            \begin {itemize} [noitemsep]
                \item Adaptabilidade
                \item Colaboración e Traballo en equipo
                \item Resolución de Problemas
                \item Independenza
                \item Curiosidade
            \end {itemize}
        \end {minipage}
    \end {minipage}
    \end {flushright}

    %%%%%%%%%%%%%%%%%%%%%%
    %%%%% LANGUAGES %%%%%%
    %%%%%%%%%%%%%%%%%%%%%%

    % title & separator
    %\sectiontitle {languages}
    \begin {flushleft}
    \begin {minipage}[c]{\leftcolumwidth\textwidth}
        \begin {flushright}
        \!\MakeUppercase {linguas}
        \hspace* {10px}
        \end {flushright}
    \end {minipage}
        \begin {tikzpicture}
            \hspace{-4px}
            \draw [line width=\sectionlinethickness, namelines] (1,0) -- (10.5,0);
        \end {tikzpicture}
    \end {flushleft}


    % section contents
    \begin {flushright}
    \begin {minipage} [t] {\rightcolumwidth\textwidth}
        \begin {minipage} [t] {0.7\textwidth}
            \begin {tabular} {llcccccc}
                Lingua & Acreditación & A1 & A2 & B1 & B2 & C1 & C2 \\[2px]
                \hline \\[-9px]
                Español & Nativo & \newmoon & \newmoon & \newmoon & \newmoon & \newmoon & \newmoon \\
                Galego & Nativo & \newmoon & \newmoon & \newmoon & \newmoon & \newmoon & \newmoon \\
                Inglés & Cambridge & \newmoon & \newmoon & \newmoon & \newmoon & \newmoon & \fullmoon \\
                Francés & DELF & \newmoon & \newmoon & \newmoon & \fullmoon & \fullmoon & \fullmoon \\
            \end{tabular}
        \end {minipage}
        \hfill
        \begin {minipage} [t] {0.2\textwidth}
        \vspace* {-30px}
        \hspace* {12px}
            \includesvg [width=60px, height=60px] {qr_code}
        \end {minipage}
    \end {minipage}
    \end {flushright}


\end{document}
